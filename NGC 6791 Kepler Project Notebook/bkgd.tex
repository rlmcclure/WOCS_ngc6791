\chapter*{Background}
\section*{Beginning the Project}
 Met Melinda Soares in Fall of 2019 and she suggested we apply her K2 method [link paper] to cluster in the OG Kepler Mission field. NGC 6791 is a remarkable 8 Gyr old, populous, metal-rich open cluster. Matieu group has initial RV studies of evolved stars by Tofflemire and Milliman. Motivation of the study is the Kepler search for blue stragglers, requiring faint RV observations on main sequence. 

 \textbf{Include images of the field, the cluster, etc here.}

\subsubsection{AAS Summer 2020 Abstract}
 The original Kepler mission observations included a super-stamp of the open cluster NGC 6791, a cluster of particular interest as one of the oldest and most metal rich open clusters in the Milky Way. We present initial results of a forthcoming catalog of 100ppm or better photometric-precision light-curves of stars in the NGC 6791 superstamp, along with the surrounding field Kepler Objects of Interest postage stamps.  We apply an image subtraction reduction pipeline useful for crowded fields of the core of open clusters, as used for K2 superstamps by Soares-Furtado et al. 2017. This work will provide light curves and analyses of all identified variable stars in the cluster. In particular, we intend to use these data to search for only the second population of blue lurkers.  We also anticipate unforeseen discoveries in this benchmark open cluster due to the precision of the light curves and quantity of resolved objects.  We have underway a WIYN Open Cluster Study radial-velocity survey of the cluster to support analyses of these discovered objects to V = 18. We intend to apply this pipeline for similar analyses on the other three open cluster super-stamps in the Kepler field.

 NGC 6791 is about X pc away and spans about XX in size on the sky. Current state of the research on NGC 6791 is \textbf{???}.
 "Planets in open star clusters should be about as common as they are around isolated stars, however, but not many have been found because they are not easy to detect. Two of these open cluster planets were found with the Kepler instrument. These discovered transiting open cluster planets are Neptune-sized and were given the names Kepler-66b and Kepler-67b. Both were found in the 1 billion year old open cluster NGC 6811 (Meibom, 2013)."  \\ \\ 

 "Given that the stars in stellar clusters are coeval and share a common metallicity, these sites are extraordinary laboratories to investigate both the quantity of planets forming in dense environments and the types of stars they tend to orbit.\\ 
 How do we examine dense, crowded stellar regions when the light is extremely blended and most other analysts have chosen to avoid them entirely?  The key will be employing an image subtraction technique developed by Alard and Lupton. This technique has been employed on the HAT data set and, as such, Chelsea and Joel know a lot about how it works. Using image subtraction, dense stellar regions will become much sparser and much of the blended light can be removed. Using the subtracted image,  I will search open cluster regions for the signals of transiting exoplanets and stellar variability."\\ 

\subsection*{Notes about collaborators}
\subsubsection{Melinda Soares}

\subsubsection{Isabel Collman}

\subsubsection{Joel Hartman}
 Joel Hartman was a very helpful resource in Melinda's iteration of this project. He  works on HATNet and HatSouth transient surveys. He searches for variable stars, including transiting planets orbiting bright stars. He created an open-source light curve analysis program called \texttt{VARTOOLS}, which includes methods for calculating variability and periodicity statistics of light curves. It is less visually intuitive than Waqas Bhatti's python-written software \texttt{Astrobase}. Both options are available.
